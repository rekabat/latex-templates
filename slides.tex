% --------------------------------------------------------------
%                       Initialize
% --------------------------------------------------------------

\documentclass{beamer}


\usepackage{mathHW}
% contains special macros for math environment:
%     unit, p, set, abs, floor, ceil, vec, norm, inner, kernel
% contains special environments for homework:
%     hwproblem, hwsolution
% contains special macros for homework:
%     hwprob, hwanswer, hwplaceholder
% contains useful all-purpose macros:
%     n, lessSeparation, append



% --------------------------------------------------------------
%                    Beginning of Document
% --------------------------------------------------------------

\title[Short Title]{The Very Long Title that may or may not Contain the Short Title}
\author{Troy Baker}
\institute{University of Florida}
\date{April 14, 2015}

\usetheme{CambridgeUS}
\usecolortheme{dolphin}
\begin{document}

% --------------------------------------------------------------
%                           Title
% --------------------------------------------------------------

\beamertemplatenavigationsymbolsempty

\setbeamertemplate{itemize items}[circle]
\setbeamertemplate{enumerate items}[circle]

\setbeamertemplate{section in toc}[circle]
% \setbeamertemplate{subsection in toc}[circle]
\AtBeginSubsection[]
{
  \begin{frame}<beamer>
    \frametitle{Layout}
    \tableofcontents[currentsection,currentsubsection]
  \end{frame}
}

% --------------------------------------------------------------
%                           Body
% --------------------------------------------------------------


\begin{frame}
    \titlepage
\end{frame}

\section{Introduction}
\subsection{Definitions}
\begin{frame}{Introduction}
    This is a short introduction to Beamer class.
    \pause
    This is some more.
\end{frame}


\begin{frame}{Blocks}
    \begin{block}{Block title}
        This is a block in blue
    \end{block}

    \begin{definition}
        Definition
    \end{definition}

    \begin{theorem}
        Theorem
    \end{theorem}

    \begin{proof}
        Proof
    \end{proof}
\end{frame}


\begin{frame}{Test}
    \begin{itemize}
        \item<2-> appears from slide 2 on
        \begin{itemize}
            \item Sub list
        \end{itemize}
        \item<2-4> appears from slides 2-4
        \item<4> appears on slide 4
        \item<5-> appears from slide 5 on
    \end{itemize}

    \uncover<4-5>{
        Appear from slides 4-5
    }

    \invisible<2>{HI!} \alt<2>{BYE!}{oops}
    \alert<2>{Red words on slide 2.} \color<2>{green}{Green words on slide 2.} %remove #s to make always alert or green


    \begin{itemize}
        \item Language used by Beamer: L\uncover<2->{A}TEX
        \item Language used by Beamer: L\only<2->{A}TEX
    \end{itemize}

    \begin{enumerate}[<+-| alert@+>]
        \item L
        \item A
        \item T
        \item E
        \item X
        \item !
        \begin{enumerate}
            \item Sub items
        \end{enumerate}
    \end{enumerate}
\end{frame}

\subsection{Sub nothing}
\begin{frame}{What?}
\end{frame}

\section{Nothing}
\begin{frame}{do?}
\end{frame}

\subsection{Sub nothing2}
\begin{frame}{you want?}
\end{frame}


% --------------------------------------------------------------
%                        End of Body
% --------------------------------------------------------------

\end{document}

% --------------------------------------------------------------
%                       End of Document
% --------------------------------------------------------------
